\documentclass{article}

\usepackage{fullpage,amsmath}
\usepackage{algorithm,algpseudocode}



\begin{document}

\section{Key ideas}


\section{Basic Rules}

Vector Algebra rules:

Matrix Algebra rules:

\begin{align}
    AB \ne BA
\end{align}

\begin{align}
    & A+B = B+A \\
    & A+B+C = A+(B+C) = (A+B)+C \\
    & A(BC)=A(BC)=(AB)C \\
    & A(B+C) = AB+ AC \\
\end{align}

Matrix transposition rules:
\begin{align}
    & (A^T)^T = A \\
    & (A+B)^T = A^T+B^T \\
    & (AB)^T = B^T A^T \\
    & (ABC)^T = C^T B^T A^T 
\end{align}

Matrix inverse rules:

\begin{align}
    &  AI = IA = A \\
    &  A0 = 0 \\
    &  I = A^{-1} A = A A^{-1} \\
    &  (A^{-1})^{-1} = A \\
    &  (AB)^{-1} = B^{-1} A^{-1}\\
    &  (ABC)^{-1} = C^{-1} B^{-1} A^{-1} \\
    &  (A^T)^{-1} = (A^{-1})^T 
\end{align}


\section{Gram-Schmidt algorithm}

\begin{algorithm}
    \caption{Gram-Schmidt algorithm}
    \begin{algorithmic}
        \State \textbf{given} n-vectors $a_1,...,a_k$
        \For{$i=1,...,k$}
        \State Orthogonalization. $\tilde{q}_i = a_i - (q_1^T a_i) q_1 - \cdots - (q_{i-1}^T a_i) a_{i-1} $
        \State Test for linear dependence. if $\tilde{q}_i = 0$, quit.
        \State Normalization. $q_i = \tilde{q}_i / ||\tilde{q}_i||$
        \EndFor
    \end{algorithmic}
\end{algorithm}

This algorithm has following properties
\begin{enumerate}
    \item $\tilde{q}_i \ne 0$, so the linear dependence test in step 2 is not satisfied, and there will not be divide by 0 error in step 3.
    \item $q_1,...,q_i$ are orthonormal to each other.
    \item $a_i$ is a linear combination of $q_1,...,q_i$.
    \item $q_i$ is a linear combination of $a_1,...,a_i$.
\end{enumerate}



\end{document}


\ref{Introduction to Applied Linear Algebra: Vectors, Matrices, and Least Squares - Boyd and Vandenberghe}
