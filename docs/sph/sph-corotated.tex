
\documentclass{article}

\usepackage{fullpage,amsmath,amssymb}
\newcommand{\dt}{\Delta t}
\newcommand{\pdv}[2]{\frac{\partial #1}{\partial #2}}
\newcommand{\dv}[2]{\frac{d #1}{d #2}}
\newcommand{\gp}[1]{{\left({#1}\right)}}
\newcommand{\ab}[1]{{\left|{#1}\right|}}
\newcommand{\mx}[1]{\begin{pmatrix}#1\end{pmatrix}}
\newcommand\numberthis{\addtocounter{equation}{1}\tag{\theequation}}

\newcommand{\xx}{\mathbf{x}}
\renewcommand{\AA}{\mathbf{A}}
\newcommand{\BB}{\mathbf{B}}
\newcommand{\MM}{\mathbf{M}}
\newcommand{\nn}{\mathbf{n}}
\newcommand{\FF}{\mathbf{F}}
\newcommand{\GG}{\mathbf{G}}
\newcommand{\bg}{\mathbf{g}}
\newcommand{\PP}{\mathbf{P}}
\newcommand{\pp}{\mathbf{p}}
\newcommand{\QQ}{\mathbf{Q}}
\newcommand{\rr}{\mathbf{r}}
\newcommand{\II}{\mathbf{I}}
\newcommand{\KK}{\mathbf{K}}
\newcommand{\uu}{\mathbf{u}}
\newcommand{\vv}{\mathbf{v}}
\newcommand{\ut}{\mathbf{\tilde{u}}}
\newcommand{\ff}{\mathbf{f}}
\newcommand{\cc}{\mathbf{c}}
\newcommand{\ee}{\mathbf{e}}
\newcommand{\CC}{\mathbf{C}}
\newcommand{\DD}{\mathbf{D}}
\newcommand{\JJ}{\mathbf{J}}
\newcommand{\dd}{\mathbf{d}}
\newcommand{\HH}{\mathbf{H}}
\newcommand{\EE}{\mathbf{E}}
\newcommand{\kk}{\mathbf{k}}
\newcommand{\zz}{\mathbf{z}}
\newcommand{\ww}{\mathbf{w}}
\newcommand{\bb}{\mathbf{b}}
\newcommand{\bmu}{\boldsymbol{\mu}}
\DeclareMathOperator{\tr}{tr}

\begin{document}

\section{SPH-corotated}

Reference \cite{becker:2009:sph-corotated}

Let $\uu_{ij}=\uu_i - \uu_j = (\xx_i - \xx_i^0) - (\xx_j - \xx_j^0) $ be displacement vector. The Jacobian of mapping $\xx^0 \rightarrow \xx_0 + \uu$ is $\JJ = \nabla \xx^0 + \nabla \uu^T = \II + \nabla \uu^T$

The SPH approximation of gradient $\nabla \uu_i$  will be
\begin{equation}
  \nabla \uu_i = \sum_j \tilde{v}_i \uu_{ji} \nabla W(\xx_{ij}^0 , h)^T \\
\end{equation}

There are two types of strain tensor introduced used in this paper. One is the non-linear Green-Saint-Venant strain tensor
\begin{equation}
\epsilon = \frac{1}{2} (\JJ^T \JJ - \II)
\end{equation}

Another is the linear Cauchy-Green strain tensor.  Both tensor referenced to \cite{bathe:2006:fep}. 
\begin{equation}
\epsilon  = \frac{1}{2} (\JJ + \JJ^T) - \II
\end{equation}

It has to meet the condition $\nabla \uu  \nabla \uu^T  = 0 $
\begin{align}
  \epsilon &= \frac{1}{2} ((\II + \nabla \uu) (\II + \nabla \uu^T) - \II ) \\
  &= \frac{1}{2} ( \II + \nabla \uu + \II + \nabla \uu^T + \nabla \uu  \nabla \uu^T  - 2\II) \\
  &= \frac{1}{2} (\JJ + \JJ^T) - \II
\end{align}

The strain energy $U_i$ is 
\begin{equation}
  U_i = \tilde{v}_i \frac{1}{2} (\epsilon_i \sigma_i) = \tilde{v}_i \frac{1}{2} (\epsilon_i \CC \epsilon)
\end{equation}

The $\CC \in \mathbb{R}$ is characterized by Young modulus $E$ and the Poisson ration $\nu$.

with $\tilde{v}_i$ to be the initial volume of particle $i$.
\begin{equation}
\tilde{v}_i = \frac{m_i}{\sum_j m_j W(\xx_{ij}^0,h)} = \frac{m_i}{\rho_{i0}}
\end{equation}

%% \begin{align}
%%   \nabla_{\uu_j} U_i & \\
%%   U_i &= \tilde{v}_i \sigma_i          \frac{1}{2} (\frac{1}{2} (\JJ^T \JJ - \II)) \\
%%   \nabla_{\uu_j} U_i &=\tilde{v}_i \sigma_i (- ( \II + \nabla \uu_i^T)  \nabla W(\xx_{ij}^0,h)
%% \end{align}
%% TODO I don't understand this part of gradient U
%% I also don't find the linear


The elastic forces $\ff_{ji}$ exerted on particle $j$ by particle $i$ when using the nonlinear Green-Saint-Venant strain tensor is
\begin{equation}
  \ff_{ji} = - \nabla_{\uu_j} U_i = - \tilde{v}_i ( \II + \nabla \uu_i^T) \sigma_i \tilde{v}_j \nabla W(\xx_{ij}^0,h)
\end{equation}

When using Cauchy-Green tensor, it will simplifies to 
\begin{equation}
  \ff_{ji} = - \tilde{v}_i \sigma_i \tilde{v}_j \nabla W(\xx_{ij}^0,h)
\end{equation}



\bibliographystyle{apalike}
\bibliography{../../references}
\end{document}
